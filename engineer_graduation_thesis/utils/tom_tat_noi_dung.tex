
\def \tomtatnoidung{
    \section*{Tóm tắt nội dung đồ án}
    \addcontentsline{toc}{section}{Tóm tắt nội dung đồ án}
    Cách mạng công nghiệp 4.0 đã mang đến cho con người một kỷ nguyên khai phá dữ liệu. Điều này đặt ra bài toán không chỉ làm sao để khai phá được dữ liệu một cách hiệu quả mà còn làm sao để sinh ra được thêm nhiều dữ liệu một cách tự động và với số lượng lớn. Do đó, trong khuôn khổ của đồ án tốt nghiệp, em sẽ nghiên cứu về ý tưởng chung, kiến trúc, hàm loss, phương pháp đánh giá, những vấn đề tồn đọng và cách giải quyết các vấn đề đó của mô hình Generative Adversarial Networks (GANs) tổng quát nhằm giải quyết bài toán sinh dữ liệu ảnh nói và mô hình Multimodal Unsupervised Image-to-Image Translation (MUNIT) giúp giải quyết bài toán sinh dữ liệu ảnh mới từ ảnh đã có sẵn (Image-to-Image Translation). Bên cạnh những kết quả đã được công bố trong bài báo, em đã thử nghiệm mô hình với bộ dữ liệu khác để chứng minh tính đúng đắn của mô hình và lấy đó làm tiền đề cho việc phát triển những mô hình phức tạp hơn trong tương lai.
    \vskip 2.5in

    \hspace{7.7cm}
    \normalsize {Hà Nội, ngày \hspace{0.5cm} tháng \hspace{0.5cm} năm}
    \vskip 0in

    \hspace{8.5cm}
    \normalsize {\textit{Sinh viên thực hiện}}
}