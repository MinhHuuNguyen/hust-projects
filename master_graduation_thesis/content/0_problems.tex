\def\problems{
    \section{Phát biểu các bài toán}
    \subsection{Bài toán object detection}
    Bài toán object detection là một bài toán rất phổ biến trong computer vision và được coi là một trong số các bài toán machine learning kinh điển.
    Tính ứng dụng của bài toán object detection trong thực tiễn là rất lớn trong nhiều ngành nghề khác nhau.
    Object detection được sử dụng trong y tế giúp xác định vị trí bị bệnh trong cơ thể, trong bảo mật giúp định vị vị trí của con người trong những khu vực cấm, trong nông nghiệp giúp xác định số lượng nông sản ...
    Ngoài ra, một ứng dụng của object detection đang thu hút rất nhiều sự quan tâm hiện nay của nhiều tập đoàn công nghệ lớn, đó là định vị trong hệ thống xe tự hành.

    \noindent
    Bài toán object detection là sự tổng hợp của hai bài toán con: \textit{object localization} và \textit{image classification}.
    Cụ thể hơn, bài toán object localization là bài toán định vị vị trí của object trong ảnh, mô hình giải bài toán này nhận đầu vào là một ảnh gồm một hoặc nhiều đối tượng và trả đầu ra là một hoặc nhiều bounding box đại diện cho vị trí của từng đối tượng.
    Trong khi đó, bài toán image classification là bài toán phân lớp ảnh, mô hình giải bài toán này nhận đầu vào là một ảnh chứa một đối tượng và trả đầu ra là lớp của đối tượng đó.
    Bài toán object detection kết hợp cả hai bài toán trên và yêu cầu mô hình vừa định vị vị trí của một hoặc nhiều đối tượng trong ảnh vừa xác định lớp của từng đối tượng đó.

    \noindent
    Một số bài toán khác khá gần gũi với bài toán object detection là bài toán object segmentation.
    Tuy nhiên, trong khuôn khổ của luận văn, chúng ta sẽ không đi sâu vào bài toán này.

    \noindent
    Với sự quan tâm của giới nghiên cứu cho bài toán object detection, đã có rất nhiều các nghiên cứu và giải pháp ra đời đạt được độ chính xác cao và chạy trong thời gian thực.
    Chi tiết về một số nghiên cứu giải bài toán object detection sẽ được thảo luận kỹ hơn trong \textbf{\textit{Chương 2. Các mô hình giải quyết bài toán object detection}}.

    \subsection{Bài toán object detection trong ảnh chất lượng cao}
    Mặc dù đã có nhiều các nghiên cứu quan tâm đến bài toán object detection, nhưng vẫn tồn tại những vấn đề nan giải và đến nay vẫn là chủ đề hot, đó chính là bài toán object detection với ảnh chất lượng cao.
    Bài toán object detection với ảnh chất lượng cao thường xuất hiện khi làm việc với ảnh trích xuất từ camera an ninh hay ảnh vệ tinh.

    \noindent
    Các bộ dữ liệu thường được sử dụng trong quá trình nghiên cứu thường có kích thước khá nhỏ, thường là 112×112, 224×224 hay 512×512.
    Những bộ dữ liệu ảnh có kích thước khoảng từ 1000 đến 2000 đã là rất lớn trong quá trình nghiên cứu.
    Tuy nhiên, hiện nay, công nghệ ghi hình ngày càng phát triển kéo theo những chiếc camera hiện đại có thể lưu giữ lại được những hình ảnh có độ phân giải rất lớn như 8K (7680×4320) hay thậm chí là 16K (15360×8640).
    Việc xử lý những hình ảnh lớn này với các mô hình học sâu thật sự gây ra nhiều vấn đề về chi phí tính toán và thời gian tính toán.
    Do đó, việc sử dụng những hình ảnh chất lượng cao trong quá trình dự đoán đã khó, việc train mô hình với những hình ảnh này gần như bất khả thi.

    \noindent
    Một cách đơn giản mà ai cũng có thể đề xuất đó là thu nhỏ kích thước ảnh trước khi đưa vào mô hình học sâu.
    Tuy nhiên cách làm này gây ra việc mất mát rất nhiều thông tin của các đối tượng ở trên ảnh.
    Đối với những đối tượng có kích thước lớn trong ảnh, việc thu nhỏ ảnh ban đầu không ảnh hưởng quá nhiều đến những đối tượng này, vì sau khi thu nhỏ lại, kích thước của chúng vẫn đủ để mô hình thu thập được các đặc trưng cần thiết.
    Còn đối với các đối tượng có kích thước nhỏ, sau khi thu nhỏ ảnh ban đầu, những đối tượng này gần như biến mất khỏi ảnh và gây ra khó khăn cho mô hình để có thể thu thập được các đặc trưng của các đối tượng này.
    Hơn nữa, trong nghiên cứu \cite{singh2018analysis}, nhóm tác giả đã chứng minh rằng mô hình học sâu với nền tảng là kiến trúc CNN không thể học một cách hiệu quả cả những object có kích thước lớn và cả những object có kích thước nhỏ trong ảnh (vấn đề large-scale variance).
    Từ hai vấn đề trên, ta cần giải pháp tốt hơn để giải quyết ảnh chất lượng cao sao cho vừa đảm bảo về độ chính xác vừa đảm bảo về chi phí và thời gian tính toán của mô hình.

    \noindent
    Chi tiết về một số nghiên cứu giải bài toán object detection trong ảnh chất lượng cao sẽ được thảo luận kỹ hơn trong \textbf{\textit{Chương 3. Các mô hình giải quyết bài toán object detection trong ảnh chất lượng cao}}.

    \subsection{Bài toán face detection}
    Bài toán face detection là một bài toán nền tảng cực kỳ quan trọng cho rất nhiều các bài toán khác về khuôn mặt như face recognition - bài toán nhận diện mặt người, face tracking - bài toán theo dõi khuôn mặt, face synthesis - bài toán sinh ra ảnh khuôn mặt, face attributes classification - bài toán phân lớp các thuộc tính trên khuôn mặt ...
    Những ứng dụng của nhóm bài toán liên quan đến khuôn mặt có thể kể đến như nhận diện khách hàng, điểm danh chấm công, phân tích cảm xúc ...
    Với những tiềm năng trên, face detection trở thành một nhánh nghiên cứu thu hút rất nhiều sự quan tâm của giới nghiên cứu vì tính ứng dụng cao và động lực đẩy độ chính xác của mô hình giải bài toán này lên đến tuyệt đối.
    
    \noindent
    Nhiều nghiên cứu đã nhân mạnh vào những đặc thù riêng biệt của khuôn mặt con người so với đồ vật, cây cối hay con vật để đưa ra những giải pháp nhằm thúc đẩy độ chính xác của mô hình.
    Tuy vậy, trong nghiên cứu \cite{zhu2020tinaface}, nhóm tác giả đã chỉ ra rằng face detection vẫn chỉ là một bài toán con của bài toán object detection và bài toán face detection vẫn có thể được giải một cách hiệu quả bằng các mô hình object detection.

    \noindent
    Chi tiết về một số nghiên cứu được thiết kế đặc biệt giải bài toán face detection sẽ được thảo luận kỹ hơn trong \textbf{\textit{Chương 4. Các bộ dữ liệu và mô hình giải quyết bài toán face detection}}.
    
    \subsection{Bài toán face detection trong ảnh chất lượng cao}
    Với những điểm tương đồng giữa bài toán face detection và bài toán object detection, các mô hình giải bài toán face detection cũng gặp phải hai vấn đề lớn khi xử lý với ảnh chất lượng cao: sự xuất hiện của các khuôn mặt nhỏ trong ảnh và sự chênh lệch về kích thước giữa các khuôn mặt lớn và nhỏ trong ảnh.

    \noindent
    Hơn nữa, cho đến nay, số lượng các bộ dữ liệu giúp giải quyết bài toán face detection trong ảnh chất lượng cao vẫn chưa được chia sẻ nhiều.
    Bộ dữ liệu phổ biến nhất để giải quyết bài toán face detection là bộ dữ liệu WIDER FACE \cite{yang2016wider} mặc dù chứa những ảnh đa dạng về ánh sáng, màu sắc, góc quay, che khuất ... nhưng độ phân giải của ảnh trong bộ dữ liệu này là chưa thực sự cao, chỉ khoảng 1000 - 1500 pixel.
    Trong khi đó, yêu cầu về ảnh chất lượng cao với camera hiện đại là khoảng 4000 - 6000 pixel, thậm chí lên đến hơn 10000 pixel.

    \noindent
    Chúng ta sẽ cùng nhau nghiên cứu về mô hình giải quyết những vấn đề trên trong \textbf{\textit{Chương 5. Mô hình RetinaFocus và bộ dữ liệu WIDER FACE 4K giải quyết bài toán face detection trong ảnh chất lượng cao}}.
}