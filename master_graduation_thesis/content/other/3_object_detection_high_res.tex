\def\objectdetectionhighres{
    \section{Các mô hình giải quyết bài toán object detection trong ảnh chất lượng cao}
    Bài toán object detection trong ảnh chất lượng cao là một chủ đề nhận được nhiều sự quan tâm trong thời gian vừa qua với nhiều nghiên cứu có giá trị tập trung để giải quyết hái vấn đề lớn là độ chính xác và tốc độ của mô hình.
    Các nghiên cứu này được chia làm hai trường phái: \\
    - Nhóm 1 bao gồm các giải pháp dựa trên ý tưởng chung là chia ảnh đầu vào có kích thước lớn thành dạng grid, sau đó thiết kế một thuật toán để giúp mô hình học sâu sử dụng các ảnh từ grid và tổng hợp các kết quả đầu ra lại trở thành một kết quả chung cho ảnh đầu vào có kích thước lớn.
    Các phương pháp trong nhóm này đạt độ chính xác tương đối cao do xử lý toàn bộ ảnh đầu vào và không bỏ đi khu vực nào trên ảnh, nhưng tốc độ lại không thực sự nhanh do phải xử lý một số lượng ảnh nhỏ từ grid rất lớn. \\
    - Nhóm 2 bao gồm các giải pháp dựa trên ý tưởng chung là thiết kế một mô hình hoặc một thuật toán riêng biệt xác định vị trí nào trên ảnh đầu vào có kích thước lớn là vị trí đáng chú ý và cần tập trung.
    Từ đó, loại bỏ đi các phần chứa ít thông tin trên ảnh đầu vào.
    Độ chính xác của các phương pháp trong nhóm này phụ thuộc nhiều vào mô hình hoặc thuật toán xác định vị trí đáng chú ý.
    Tuy nhiên, tốc độ của nhóm mô hình này lại rất tốt vì chúng đã bỏ qua phần lớn các khu vực trên ảnh đầu vào.

    \noindent
    Trong khuôn khổ nghiên cứu của luận văn, ta sẽ cùng tìm hiểu kỹ hơn về lý do tại sao bài toán object detection trong ảnh chất lượng cao lại là một bài toán khó trong \textbf{\textit{phần 3.1. Các thí nghiệm về tỷ lệ giữa kích thước của đối tượng trong ảnh và kích thước ảnh}}.
    Sau đó, ta bàn luận về một số phương pháp chuẩn bị dữ liệu tiêu biểu thuộc nhóm 2 qua các \textbf{\textit{phần 3.2. Phương pháp chuẩn bị dữ liệu SNIP, phần 3.3. Phương pháp chuẩn bị dữ liệu SNIPER}}.
    Cuối cùng, ta sẽ thảo luận về một mô hình khá hoàn thiện để xử lý ảnh chất lượng cao qua \textbf{\textit{phần 3.4. Mô hình AutoFocus}}.

    \subsection{Các thí nghiệm về tỷ lệ giữa kích thước của đối tượng trong ảnh và kích thước ảnh}
    \def\highresexperiments{
    Trước khi bàn luận đến những mô hình giúp giải quyết bài toán object detection trong ảnh chất lượng cao, ta cần xem xét đến vai trò của tỷ lệ giữa kích thước của đối tượng trong ảnh và kích thước ảnh thông qua một số thí nghiệm.
    Cụ thể hơn, nhóm tác giả muốn đánh giá sự ảnh hưởng \textit{domain shift} lên kết quả của mô hình.
    \textit{Domain shift} được định nghĩa là sự thay đổi về tỷ lệ giữa kích thước của đối tượng trong ảnh và kích thước ảnh giữa tập dữ liệu train và tập dữ liệu test.
    \textit{Domain shift} rất phổ biến trong bài toán object detection, khi kích thước ảnh sử dụng trong quá trình train thường nhỏ (nhằm cải thiện tốc độ train mô hình) và kích thước ảnh sử dụng trong quá trình test thường lớn hơn (nhằm cải thiện độ chính xác của mô hình).
    Các thí nghiệm này được xây dựng và công bố kết quả bởi nhóm tác giả của mô hình SNIP \cite{singh2018analysis}.

    \noindent
    \textbf{\textit{Các thí nghiệm trong bài toán image classification}} \\
    Ba mô hình được sử dụng trong các thí nghiệm được thiết kế như sau: \\
    - Mô hình CNN-B: Đây là mô hình CNN sử dụng backbone ResNet-101 và đã được train với bộ dữ liệu ảnh có kích thước 224x224. \\
    - Mô hình CNN-S: Đây là mô hình CNN sử dụng backbone ResNet-101 và đã đã được thay đổi các tham số stride và kích thước của conv trong lớp đầu tiên đối với từng bộ dữ liệu ảnh có kích thước khác nhau khi train mô hình.
    Khi train với bộ dữ liệu ảnh có kích thước 48x48, tác giả dùng conv có kích thước 3x3 và stride bằng 1, khi train với bộ dữ liệu ảnh có kích thước 96x96, tác giả dùng conv có kích thước 5x5 và stride bằng 2. \\
    - Mô hình CNN-B-FT: Đây là mô hình CNN sử dụng backbone ResNet-101 và đã được train với bộ dữ liệu ảnh có kích thước 224x224 giống như CNN-B.
    Sau đó fine-tune lại với bộ dữ liệu ảnh có kích thước nhỏ hơn nhưng được upsample lên lại kích thước 224x224.

    \begin{figure}[H]
        \centering
        \includegraphics[width=12cm] {images/snip_models_to_exp.png}
        \caption{Ba mô hình được sử dụng trong các thí nghiệm (Nguồn: \cite{singh2018analysis})}
        \label{fig:snip_models_to_exp}
    \end{figure}

    \noindent 
    Nhóm tác giả thực hiện thí nghiệm đầu tiên là \textit{Thí nghiệm Naive Multi-Scale Inference}.
    Nhóm tác giả kiểm tra đánh giá mô hình CNN-B đơn giản với các bộ dữ liệu có kích thước ảnh khác nhau.
    Đầu tiên, nhóm tác giả thu nhỏ kích thước ảnh trong bộ dữ liệu ImageNet từ 224x224 xuống 48x48, 64x64, 80x80, 96x96, 128x128.
    Tiếp theo, nhóm tác giả tăng kích thước ảnh từ 48x48, 64x64, 80x80, 96x96, 128x128 lên lại 224x224.
    Cuối cùng, với các bộ dữ liệu mới nói trên, nhóm tác giả đưa vào mô hình CNN-B và quan sát kết quả.

    \begin{figure}[H]
        \centering
        \includegraphics[width=9cm] {images/snip_naive_multi_scale_infer}
        \caption{Kết quả của thí nghiệm Naive Multi-Scale Inference (Nguồn: \cite{singh2018analysis})}
        \label{fig:snip_naive_multi_scale_infer}
    \end{figure}

    \noindent
    \textbf{Kết quả}:
    Ảnh sử dụng trong quá trình train và trong quá trình test càng khác nhau về kích thước, khả năng của mô hình càng giảm.
    Cụ thể hơn, kết quả về top-1 accuracy của mô hình CNN-B với các bộ dữ liệu được tăng kích thước từ 48x48, 64x64, 80x80, 96x96, 128x128 lần lượt là 33.8\%, 45.8\%, 53.8\%, 59.7\%, 67.1\%.
    Trong khi đối với bộ dữ liệu được giữ nguyên kích thước 224x224, mô hình CNN-B đạt kết quả 76.8\%. \\
    \textbf{Kết luận}:
    Đối với bài toán classification, việc test mô hình với các kích thước ảnh khác mà mô hình chưa được train sẽ gây ra kết quả suy giảm đáng kể.
    Sự chênh lệch càng lớn giữa kích thước ảnh train và kích thước ảnh test càng gây ra sự suy giảm kết quả đáng kể.

    \noindent
    Nhóm tác giả thực hiện thí nghiệm thứ hai là \textit{Thí nghiệm Resolution Specific Classifiers}.
    Trong thí nghiệm này, nhóm tác giả thay đổi một chút kiến trúc của mô hình CNN-S.
    Trong kiến trúc của mô hình CNN-B, lớp conv đầu tiên có kích thước stride = 2, và sau đó là lớp max pooling với stride = 2x2.
    Do đó, ngay từ các lớp đầu tiên, các mô hình trên xoá đi các thông tin của các đối tượng nhỏ trong ảnh.
    Từ quan sát trên, nhóm tác giả thay đổi kiến trúc và tạo ra mô hình CNN-S sao cho nó đạt được kết quả tốt nhất trên bộ CIFAR10 (đây là bộ dữ liệu với kích thước ảnh nhỏ).
    Sau đó, nhóm tác giả so sánh kiến trúc của mô hình CNN-S với mô hình CNN-B.

    \begin{figure}[H]
        \centering
        \includegraphics[width=9cm] {images/snip_res_spec_cls}
        \caption{Kết quả của thí nghiệm Resolution Specific Classifiers và Fine-tuning High-Resolution Classifiers trên bộ dữ liệu ảnh có kích thước 48x48 (Nguồn: \cite{singh2018analysis})}
        \label{fig:snip_res_spec_cls}
    \end{figure}

    \noindent
    \textbf{Kết quả}:
    Mô hình CNN-S đạt kết quả tốt hơn rất nhiều so với kết quả của mô hình CNN-B khi xử lý ảnh 48x48.
    Trong khi mô hình CNN-B đạt kết quả 33.8\%, mô hình CNN-S đạt kết quả 60.38\%. \\
    \textbf{Kết luận}:
    Việc thay đổi kiến trúc của mô hình sao cho phù hợp giúp cho mô hình học ảnh có resolution nhỏ tốt hơn.

    \noindent
    Nhóm tác giả thực hiện thí nghiệm cuối cùng trong phần này là \textit{Thí nghiệm Fine-tuning High-Resolution Classifiers}.
    Việc thiết kế và train mô hình (tương như như phương án của mô hình CNN-S trong thí nghiệm Resolution Specific Classifiers) với mỗi một bộ dữ liệu có kích thước ảnh khác nhau sẽ tốn quá nhiều công sức và tài nguyên tính toán.
    Từ đó, nhóm tác giả đưa ra một giải pháp đơn giản hơn đó là chiến lược xây dựng mô hình CNN-B-FT.

    \begin{figure}[H]
        \centering
        \includegraphics[width=9cm] {images/snip_res_spec_cls_96}
        \caption{Kết quả của thí nghiệm Resolution Specific Classifiers và Fine-tuning High-Resolution Classifiers trên bộ dữ liệu ảnh có kích thước 96x96 (Nguồn: \cite{singh2018analysis})}
        \label{fig:snip_res_spec_cls_96}
    \end{figure}

    \noindent
    \textbf{Kết quả}:
    Mô hình CNN-B-FT có kết quả tốt hơn so với kết quả của mô hình CNN-S. \\
    \textbf{Kết luận}:
    Thí nghiệm trên chứng minh rằng, các trọng số của mô hình CNN-B, được train từ bộ dữ liệu ảnh có kích thước lớn, hữu ích trong việc giải quyết bộ dữ liệu ảnh có kích thước nhỏ.
    Từ đó, thay vì giảm kích thước stride của lớp conv đi 2 lần (như trong kiến trúc của mô hình CNN-S), ta nên tăng kích thước của ảnh chất lượng thấp lên lần và fine-tune lại với mô hình đã được train với bộ dữ liệu ảnh có kích thước lớn.
    Hay nói cách khác, thay vì thay đổi kiến trúc của mô hình để xử lý riêng với bộ dữ liệu ảnh có kích thước nhỏ, ta nên tận dụng mô hình pretrained và fine-tune lại mô hình đó với bộ dữ liệu ảnh có kích thước mong muốn.

    \noindent
    \textbf{\textit{Các thí nghiệm trong bài toán object detection}} \\
    Hiệu năng giữa các mô hình bị ảnh hưởng bởi sự khác nhau về kích thước của ảnh trong bộ dữ liệu train và bộ dữ liệu test đã được nhóm tác giả chứng minh ở các thí nghiệm trên.
    Tuy nhiên, trong thực tế, không phải lúc nào ta cũng có thể train và test mô hình với bộ dữ liệu ảnh có cùng kích thước, đặc biệt đối với bài toán object detection.
    Lý do vì những vấn đề liên quan đến bộ nhớ của GPU cũng như là tốc độ train, do đó, thông thường kích thước ảnh trong bộ dữ liệu sử dụng trong quá trình train thường nhỏ hơn so với kích thước ảnh trong quá trình test.

    \noindent
    Trong các thí nghiệm ở phần này, nhóm tác giả sử dụng chung mô hình là Deformable - RFCN \cite{dai2017deformable}, train với những bộ dữ liệu khác nhau nhưng cùng test trên bộ dữ liệu có kích thước 1400x2000 nhằm mục đích đánh giá hiệu năng trong việc xác định các đối tượng nhỏ (kích thước nhỏ hơn 32x32).

    \noindent
    Nhóm tác giả thực hiện thí nghiệm đầu tiên là \textit{Thí nghiệm train các mô hình với kích thước ảnh khác nhau}.
    Trong thí nghiệm này, nhóm tác giả train mô hình với hai bộ dữ liệu lần lượt có kích thước ảnh là 800x1400 và 1400x2000 (tác giả lần lượt gọi là ${800}_{all}$ và ${1400}_{all}$).
    Và toàn bộ các đối tượng của cả hai bộ dữ liệu này đều được sử dụng để train. \\
    \textbf{Kết quả}:
    Hiển nhiên, ${1400}_{all}$ hoạt động tốt hơn so với ${800}_{all}$.
    Tuy nhiên, phần hơn của ${1400}_{all}$ là chưa đáng kể. \\
    \textbf{Kết luận}:
    Việc train ${1400}_{all}$ với ảnh có kích thước lớn hơn giúp các đối tượng nhỏ được tăng kích thước so với ${800}_{all}$, từ đó, các đối tượng này được train một cách dễ dàng hơn.
    Nhưng ở chiều ngược lại, những đối tượng trung bình và lớn trong ảnh cũng bị tăng kích thước theo khiến cho chúng trở nên quá lớn và trở nên khó khăn cho ${1400}_{all}$ có thể học một cách chính xác.
    Lượng dữ liệu khó học từ đối tượng trung bình và lớn này khiến cho ${1400}_{all}$ không thể vượt trội so với ${800}_{all}$.

    \begin{figure}[H]
        \centering
        \includegraphics[width=15cm] {images/snip_model_compare}
        \caption{Các cách train mô hình khác nhau 1. ${800}_{all}$, 2. ${1400}_{<80px}$, 3. MST, 4. SNIP (Nguồn: \cite{singh2018analysis})}
        \label{fig:snip_model_compare}
    \end{figure}

    \noindent
    Tiếp theo, nhóm tác giả thực hiện \textit{Thí nghiệm train mô hình với các đối tượng có kích thước cụ thể}.
    Trong thí nghiệm này, nhóm tác giả vẫn train mô hình với bộ dữ liệu có kích thước 1400x2000.
    Tuy nhiên, nhóm tác giả loại bỏ hoàn toàn các đối tượng trung bình và lớn (các đối tượng có kích thước lớn hơn 80 trong ảnh gốc) và gọi là ${1400}_{<80px}$ \\
    \textbf{Kết quả}:
    Kết quả của ${1400}_{<80px}$ rất tệ so với kết quả của ${800}_{all}$ và ${1400}_{all}$. \\
    \textbf{Kết luận}:
    Việc loại bỏ hoàn toàn các đối tượng trung bình và lớn khiến cho lượng dữ liệu mà ${1400}_{<80px}$ được học giảm đi rất nhiều so với các mô hình trên (giảm khoảng 30\% lượng biến thể của đối tượng).
    Từ đó, kết quả của ${1400}_{<80px}$ bị ảnh hưởng nghiêm trọng.
    Chúng ta cũng có thể rút ra được một kết luận là, cho dù các ${1400}_{all}$ và ${800}_{all}$ không hoàn toàn học được các đối tượng trung bình và lớn nhưng lượng dữ liệu từ các đối tượng này vẫn đóng vai trò quan trọng trong việc giúp các mô hình xác định các đối tượng nhỏ.

    \begin{figure}[H]
        \centering
        \includegraphics[width=13cm] {images/snip_results_1}
        \caption{So sánh kết quả trên chỉ số mAP của các mô hình ${1400}_{<80px}$, ${800}_{all}$, ${1400}_{all}$, MST và SNIP trên các đối tượng có kích thước nhỏ (nhỏ hơn 32x32) (Nguồn: \cite{singh2018analysis})}
        \label{fig:snip_results_1}
    \end{figure}

    \noindent
    Cuối cùng, nhóm tác giả thực hiện \textit{Thí nghiệm train với bộ dữ liệu ảnh có các kích thước khác nhau}.
    Nhóm tác giả train mô hình với bộ dữ liệu ảnh có các kích thước khác nhau được lựa chọn ngẫu nhiên trong suốt quá trình train (Multi-Scale Training) và gọi đó là MST.
    Thí nghiệm này giúp mô hình có thể được học các kích thước khác nhau với mỗi đối tượng trong bộ dữ liệu. \\
    \textbf{Kết quả}:
    Kết quả của MST xấp xỉ so với kết quả của ${800}_{all}$. \\
    \textbf{Kết luận}:
    Việc chuẩn bị dữ liệu ảnh có các kích thước khác nhau như MST khiến mô hình gặp khó khăn trong việc học các đối tượng rất lớn hoặc rất nhỏ.
    Điều này phần nào đó tương tự với hiện tượng mà ${1400}_{all}$ gặp phải.
}
    \highresexperiments

    \subsection{Phương pháp chuẩn bị dữ liệu SNIP}
    \def\snip{
    Từ những thí nghiệm trong \textit{phần 3.1}, nhóm tác giả đã đưa ra đề xuất phương pháp chuẩn bị dữ liệu Scale Normalization for Image Pyramids (gọi tắt là SNIP) \cite{singh2018analysis}.
    Phương pháp SNIP hướng đến việc sử dụng tối đa sự đa dạng trong biến thể của đối tượng trong bộ dữ liệu trong khi có thể thu hẹp được sự biến động trong kích thước của các đối tượng.
    Từ đó, mô hình sẽ có nhiều dữ liệu nhất có thể và cũng sẽ có kích thước đối tượng phù hợp nhất để học.
    Ý tưởng chung của SNIP dựa trên \textit{image pyramid} đã được nhắc đến trong phần \textit{phần 2.2. Kiến trúc Feature Pyramid Networks}.

    \noindent
    \textbf{\textit{Ý tưởng của phương pháp chuẩn bị dữ liệu SNIP}} \\
    Một điểm yếu của việc chuẩn bị dữ liệu MST ở thí nghiệm \textit{phần 3.1} là việc một ảnh bất kỳ được huấn luyện với nhiều kích thước khác nhau khiến cho khi ảnh đó ở kích thước lớn (1400x2000) thì các đối tượng lớn sẽ trở nên khó khăn cho mô hình.
    Điều này diễn ra tương tự đối với các đối tượng nhỏ khi ảnh đó ở kích thước nhỏ (480x800).

    \begin{figure}[H]
        \centering
        \includegraphics[width=16cm] {images/snip_model}
        \caption{Chi tiết kiến trúc phương pháp chuẩn bị dữ liệu SNIP trong quá trình huấn luyện và test (Nguồn: \cite{singh2018analysis})}
        \label{fig:snip_model}
    \end{figure}

    \noindent
    phương pháp chuẩn bị dữ liệu SNIP được coi là một phiên bản nâng cấp hơn của MST.
    Ý tưởng chính của SNIP là trong quá trình huấn luyện, các đối tượng có kích thước tương đồng với bộ dữ liệu pretrained của mô hình (thông thường là 224x224) sẽ được sử dụng để huấn luyện và loại bỏ các đối tượng, mà sau khi thay đổi kích thước ảnh, trở nên quá lớn hoặc quá nhỏ.

    \noindent
    \textbf{\textit{Kết quả của mô hình khi sử dụng phương pháp chuẩn bị dữ liệu SNIP}} \\
    Kết quả của mô hình khi sử dụng phương pháp chuẩn bị dữ liệu SNIP so sánh với các thí nghiệm khác trong \textit{phần 3.1} được thể hiện ở hình \ref{fig:snip_results_1}.
    Việc sử dụng SNIP giúp mô hình đạt độ chính xác cao hơn so với các mô hình sử dụng các phương pháp chuẩn bị dữ liệu khác.

    \noindent
    Nhóm tác giả cũng so sánh kết kết quả của các mô hình sử dụng và không sử dụng phương pháp chuẩn bị dữ liệu SNIP cùng các kiến trúc backbone khác nhau.
    Các mô hình sử dụng trong thí nghiệm là D-RFCN \cite{}, Mask-RCNN \cite{}, G-RMI \cite{}, Faster-RCNN \cite{}.
    Kỹ thuật được bổ sung trong một số cấu hình là RCN \cite{}.
    Các cấu hình sử dụng phương pháp chuẩn bị dữ liệu SNIP có ký hiệu \textit{+ SNIP}.
    Chi tiết các backbone sử dụng trong mỗi cấu hình được chú thích ở cột \textit{Backbone}.
    
    \begin{figure}[H]
        \centering
        \includegraphics[width=16cm] {images/snip_results_2}
        \caption{Kết quả của các mô hình sử dụng và không sử dụng phương pháp chuẩn bị dữ liệu SNIP (Nguồn: \cite{singh2018analysis})}
        \label{fig:snip_results_2}
    \end{figure}

    \noindent
    Kết quả của mô hình D-RFCN sử dụng SNIP và backbone Ensemble đạt chỉ số cao hơn khá nhiều so với các cấu hình khác.
    Ngoài ra, các cấu hình sử dụng SNIP cũng đạt chỉ số cao hơn các cấu hình không sử dụng SNIP.
    Điều này chứng minh được vai trò của phương pháp chuẩn bị dữ liệu SNIP đối với các mô hình object detection.

    \noindent
    \textbf{\textit{Vấn đề tồn đọng của phương pháp chuẩn bị dữ liệu SNIP}} \\
    Tuy rằng đã có những cải thiện trong kết quả của các mô hình sử dụng phương pháp chuẩn bị dữ liệu SNIP, nhưng nhóm tác giả vẫn nêu ra những vấn đề tồn đọng là lãng phí chi phí tính toán trong quá trình huấn luyện.
    Cụ thể, những đối tượng có kích thước lớn và bị loại bỏ trong quá trình huấn luyện, nhưng các lớp Conv \index{lớp Conv} vẫn phải tính toán trên các phần này dẫn đến sự lãng phí tài nguyên tính toán.
}
    \snip

    \subsection{Phương pháp chuẩn bị dữ liệu SNIPER}
    \def\sniper{
    Xuất phát từ những vấn đề tồn đọng của phương pháp SNIP, nhóm tác giả \cite{singh2018sniper} đã đề xuất phương pháp chuẩn bị dữ liệu Scale Normalization for Image Pyramids with Efficient Resampling (gọi tắt là SNIPER \index{SNIPER}).
    Phương pháp chuẩn bị dữ liệu SNIPER \index{SNIPER} hướng đến việc giảm thiểu đến tối đa khối lượng tính toán dư thừa từ đó tăng tốc được quá trình huấn luyện mô hình.
    Nhóm tác giả đã thành công trong việc duy trì được kết quả mà SNIP đạt được trong khi SNIPER \index{SNIPER} chỉ cần xử lý số lượng pixels \index{pixels} ảnh bằng một phần ba số lượng mà SNIP phải xử lý.
    Điều này giúp việc huấn luyện bằng phương pháp SNIPER \index{SNIPER} tiết kiệm bộ nhớ hơn và có thể sử dụng được batch size lớn hơn trong quá trình huấn luyện.
    Mô hình object detection mà nhóm tác giả sử dụng để làm các thí nghiệm với SNIPER \index{SNIPER} là mô hình Faster R-CNN \cite{ren2015faster}.

    \noindent
    \textbf{\textit{Ý tưởng của phương pháp chuẩn bị dữ liệu SNIPER \index{SNIPER}}} \\
    Nhằm tối ưu hoá khối lượng tính toán trong quá trình huấn luyện, thay vì việc thay đổi kích thước của ảnh và chọn ra những object có kích thước phù hợp trong quá trình huấn luyện, phương pháp SNIPER \index{SNIPER} cung cấp một cơ chế tiền xử lý dữ liệu.
    Từ bộ dữ liệu gốc ban đầu, các thuật toán của SNIPER \index{SNIPER} xử lý và tạo ra bộ dữ liệu mới với các ảnh trong bộ dữ liệu được gọi là các Chip.

    \noindent
    \textbf{\textit{Định nghĩa Chip và phương pháp sinh ra Chip bằng phương pháp chuẩn bị dữ liệu SNIPER \index{SNIPER}}} \\
    Chip trong SNIPER \index{SNIPER} có thể được hiểu là một phần của bức ảnh và đây cũng là đơn vị dữ liệu sử dụng trong quá trình huấn luyện mô hình.
    Các bước sinh ra bộ Chip trong thuật toán SNIPER \index{SNIPER} như sau: \\
    - \textit{Bước 1}: Nhóm tác giả định nghĩa một danh sách các kích thước ảnh {${s}_{1}$, ${s}_{2}$, ..., ${s}_{i}$, ..., ${s}_{n}$}. \\
    - \textit{Bước 2}: SNIPER \index{SNIPER} thay đổi kích thước của từng ảnh gốc ban đầu về từng kích ${s}_{i}$ trong danh sách trên.
    Ta thu được một bộ dữ liệu ảnh mới, trong đó, mỗi ảnh trong bộ dữ liệu ảnh ban đầu tương ứng với n ảnh với các kích thước khác nhau trong bộ dữ liệu ảnh mới. \\
    - \textit{Bước 3}: Trên mỗi ảnh trong bộ dữ liệu ảnh mới, SNIPER \index{SNIPER} đặt các ô vuông có kích thước \textit{KxK} sao cho tâm của các ô vuông này cách đều nhau một khoảng \textit{d} pixels \index{pixels}. \\
    - \textit{Bước 4}: SNIPER \index{SNIPER} cắt ảnh theo các ô vuông đã đặt từ bước 3 để tạo thành các Chip. Danh sách các Chip này sẽ được lựa chọn và xử lý ở các thuật toán tiếp theo.

    \noindent
    \textbf{\textit{Phương pháp lựa chọn các positive Chips}} \\
    Từ danh sách các Chip đã được tạo ra bởi thuật toán trên, nhóm tác giả thực hiện thuật toán lựa chọn các positive Chips.
    positive Chip ở đây được hiểu là các Chip chứa các groundtruth \index{groundtruth} bounding box \index{bounding box} của object mà mô hình cần học.
    Các bước lựa chọn ra positive Chip trong phương pháp SNIPER \index{SNIPER} như sau: \\
    - \textit{Bước 1}: Nhóm tác giả định nghĩa một danh sách các khoảng kích thước bounding box \index{bounding box} phù hợp tương ứng với mỗi kích thước ảnh {${s}_{1}$, ..., ${s}_{n}$}.
    Cụ thể, với kích thước ảnh ${s}_{i}$, ta định nghĩa khoảng kích thước bounding box \index{bounding box} phù hợp ${R}^{i} = [{r}_{min}^{i}, {r}_{max}^{i}]$. \\
    - \textit{Bước 2}: SNIPER \index{SNIPER} kiểm tra từng bounding box \index{bounding box} trong từng Chip trên từng kích thước ảnh.
    Những bounding box \index{bounding box} có kích thước nằm trong khoảng kích thước phù hợp ${R}^{i}$ được thêm vào danh sách các bounding box \index{bounding box} hợp lệ ${G}^{i}$. \\
    - \textit{Bước 3}: SNIPER \index{SNIPER} lựa chọn các Chip từ danh sách các Chip ở thuật toán trên sao cho số lượng bounding box \index{bounding box} hợp lệ và nằm hoàn toàn trong một Chip là nhiều nhất.
    Danh sách các positive Chips trên mỗi ảnh trong cùng một kích thước được gọi là ${C}_{pos}^{i}$.

    \begin{figure}[H]
        \centering
        \includegraphics[width=13cm] {images/sniper_pos_chip}
        \caption{Kết quả thuật toán lựa chọn các positive chips (Nguồn: \cite{singh2018sniper})}
        \label{fig:sniper_pos_chip}
    \end{figure}

    \noindent
    Một groundtruth \index{groundtruth} bounding box \index{bounding box} bất kỳ sẽ luôn có một Chip nào đó chứa nó và có thể xuất hiện ở nhiều hơn một Chip trong cùng một kích thước ảnh hoặc trong nhiều kích thước ảnh khác nhau.
    Các groundtruth \index{groundtruth} bounding box \index{bounding box} không nằm hoàn toàn trong một chip sẽ bị cắt.
    Phần cắt nếu thoả mãn khoảng kích thước phù hợp sẽ được coi là hợp lệ, phần cắt nếu không thoả mãn khoảng kích thước phù hợp sẽ được coi là vi phạm nhưng vẫn được sử dụng để gán label cho các khu vực được đề xuất bởi mô hình RPN. \\
    Mỗi groundtruth \index{groundtruth} bounding box \index{bounding box} tìm được kích thước Chip phù hợp để học, và hơn nữa kích thước của mỗi Chip nhỏ hơn rất nhiều so với mỗi kích thước ảnh tương ứng của Chip đó.
    Điều này mang lại sự tiết kiệm trong cả nguồn bộ nhớ lẫn tốc độ tính toán. \\
    Trong hình \ref{fig:sniper_pos_chip}, từ ảnh và các groundtruth \index{groundtruth} bounding box \index{bounding box} ban đầu (ảnh bên phải), thuật toán của SNIPER \index{SNIPER} đã cắt ra được các Chip (ảnh bên trái), mỗi Chip sẽ tương ứng chứa các groundtruth \index{groundtruth} bounding box \index{bounding box} hợp lệ (các bounding box \index{bounding box} màu xanh) và các groundtruth \index{groundtruth} bounding box \index{bounding box} không hợp lệ (các bounding box \index{bounding box} màu đỏ).
    Kết quả trên của SNIPER \index{SNIPER} đảm bảo tất cả các groundtruth \index{groundtruth} bounding box \index{bounding box} đều được huấn luyện ở một Chip nào đó phù hợp, trong khi kích thước của Chip chỉ bằng 1/10 kích thước của ảnh phù hợp cho groundtruth \index{groundtruth} bounding box \index{bounding box}.

    \noindent
    \textbf{\textit{Phương pháp lựa chọn các negative Chips}} \\
    Việc xây dựng thuật toán lựa chọn các positive Chips chứa tất cả các groundtruth \index{groundtruth} bounding box \index{bounding box} đã giúp cải thiện rất nhiều tốc độ huấn luyện mô hình.
    Nhưng vẫn còn một vấn đề tồn đọng, đó chính là việc thiếu hụt một lượng lớn background \index{background} (do phần lớn background \index{background} đã bị loại bỏ trong thuật toán sinh ra các positive Chips).
    Điều này dẫn đến việc mô hình thường xuyên dự đoán background \index{background} thành các vùng chứa các đối tượng.
    Do đó, nhóm tác giả đề xuất bổ sung thêm các Chip là background \index{background} không chứa đối tượng, gọi là các negative Chips. \\
    Nhóm tác giả nhận xét rằng có một số phần background \index{background} trong ảnh rất dễ để phân biệt và sẽ tốn kém chi phí tính toán để xử lý những phần background \index{background} dễ này.
    Từ nhận xét trên, nhóm tác giả đề xuất việc sử dụng các khu vực được đề xuất từ mô hình Region Proposal Network của Faster R-CNN bởi vì các khu vực này có khả năng khiến cho mô hình nhầm lẫn giữa đối tượng cụ thể nào đó và background \index{background}.

    \begin{figure}[H]
        \centering
        \includegraphics[width=13cm] {images/sniper_neg_chip}
        \caption{Kết quả thuật toán lựa chọn các negative chips (Nguồn: \cite{singh2018sniper})}
        \label{fig:sniper_pos_chip}
    \end{figure}

    \noindent
    Các bước lựa chọn negative Chips trong phương pháp SNIPER \index{SNIPER} như sau: \\
    \textit{Bước 1}: SNIPER \index{SNIPER} huấn luyện RPN của mô hình Faster R-CNN một vài epoch với chỉ các positive Chips. \\
    \textit{Bước 2}: SNIPER \index{SNIPER} sử dụng mô hình RPN đã được huấn luyện từ bước 1 để thực hiện dự đoán các khu vực đề xuất trong bộ dữ liệu huấn luyện (ảnh trong bộ dữ liệu huấn luyện lúc này cũng được thay đổi kích thước {${s}_{1}$, ..., ${s}_{n}$} như thuật toán lựa chọn positive Chips).
    Các khu vực không được RPN dự đoán sẽ được coi là những khu vực background \index{background} dễ học và loại các khu vực đó ra khỏi quá tình lựa chọn negative Chips. \\
    \textit{Bước 3}: Từ những khu vực mà RPN dự đoán, đầu tiên, SNIPER \index{SNIPER} loại các khu vực nằm trong danh sách positive Chips.
    Các Chips sẽ được lựa chọn nếu chứa nhiều hơn M khu vực được đề xuất tạo thành danh sách các negative Chips ${C}_{neg}^{i}$. \\
    \textit{Bước 4}: Danh sách các negative Chips được kết hợp cùng với danh sách các positive Chips để tiếp tục huấn luyện mô hình Faster R-CNN.

    \noindent
    \textbf{\textit{Kết quả của phương pháp chuẩn bị dữ liệu SNIPER \index{SNIPER}}} \\
    Theo thống kê của nhóm tác giả trên bộ dữ liệu COCO, bằng các phương pháp sinh và lựa chọn Chips, số lượng pixels \index{pixels} mà SNIPER \index{SNIPER} xử lý trong quá trình huấn luyện chỉ hơn khoảng 30\% so với số lượng pixels \index{pixels} cần xử lý khi huấn luyện với chỉ duy nhất một kích thước của ảnh.
    So sánh với việc huấn luyện với dữ liệu ảnh với 3 kích thước khác nhau cho mỗi ảnh, việc sử dụng các Chips có kích thước giống nhau giúp SNIPER \index{SNIPER} tối ưu về batch size hơn khi huấn luyện mô hình.
    So sánh với chiến lược của SNIP, SNIPER \index{SNIPER} đạt được độ chính xác tương đương trong khi giảm được 3 lần số pixels \index{pixels} cần phải huấn luyện trên bộ dữ liệu COCO.

    \noindent
    \textbf{\textit{Vấn đề tồn đọng của phương pháp chuẩn bị dữ liệu SNIPER \index{SNIPER}}} \\
    SNIPER \index{SNIPER} đã phần nào đó cải thiện một cách đáng kể vấn đề về thời gian huấn luyện của SNIP.
    Tuy nhiên, trong quá trình dự đoán, SNIPER \index{SNIPER} vẫn cần phải sử dụng phương pháp Image Pyramids để đạt được độ chính xác cao.
    Đối với những bộ dữ liệu ảnh có kích thước lớn thì tốc độ dự đoán của SNIPER \index{SNIPER} vẫn rất chậm và đây là vấn đề tồn đọng thúc đẩy các nghiên cứu giải quyết triệt để.
}
    \sniper

    \subsection{Mô hình AutoFocus}
    \input{content/chapter_3/3_4_auto_focus}
    \autofocus
}